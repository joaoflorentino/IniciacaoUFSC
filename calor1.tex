Em termodinâmica, calor é o processo de transferência de energia em que esta é transferida de um sistema para outro devido ao contato térmico e a uma diferença de temperatura. O termo energia térmica é utilizado para definir a energia de um corpo que aumenta com a temperatura. Temperatura é uma medida da energia interna ou entalpia e está associada ao grau de agitação das moléculas.

A partir deste conceito, vamos inferir os termos matemáticos que serão necessários para nosso propósito de elaborar as Equações de Transferência.

Passemos a estudar como se dá a condução térmica entre materiais pois isso vem ser a essência de nosso sistema em estudo. 
