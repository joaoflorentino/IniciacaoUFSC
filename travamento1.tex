\justifying
Em trabalhos de excitação de estados quânticos atômicos o travamento em frequência de lasers em diversos comprimentos de onda, expandindo o espectro controlável do infravermelho ao ultravioleta, é importante, pois possuem larguras de ressonância da ordem de alguns Kilohertz. Para isso, experimentos em física atômica, tanto em resfriamento como aprisionamento \cite{PhysRevA.90.023413}, bem como a obtenção de estados degenerados bosônicos e fermiônicos utilizam espectroscopias de absorção saturada de transições atômicas. Com isso é possível gerar um sinal dispersivo a partir de uma espectroscopia de absorção saturada por polarização \cite{4photonarxiv}, ou utilizando métodos mais complexos como a de modulação e demodulação de bandas laterais de picos de absorção. Um problema fundamental neste tipo de estabilização é a restrita modulação de frequências de apenas algumas centenas de Megahertz ao redor das ressonâncias, obtido com ajuda de moduladores Acusto-ópticos ou de cavidades ultra estáveis. Contudo, o interesse em controlar a frequência do laser ao redor da ressonância em alguns Gigahertz é crescente. Para isso utilizamos o regime de alto campo magnético, também chamado de Paschen-Back\cite{PhysRevA.93.043854}, que para altos campos magnéticos separa os níveis de energia finos do estado atômico.\par
Para o confiável funcionamento de um aparato experimental que em como pano de fundo a estabilização ativa de algum parâmetro, é de extrema importância a utilização de circuitos eletrônicos ou de \textit{feedback} que corrigem em tempo real o desvio de certo sinal alvo de um sinal de erro \cite{RevModPhys.77.783}. O melhor exemplo de um dispositivo para esta finalidade é a de um circuito analógico conhecido como PID, onde P é a parte proporcional, I a parte Integradora e D a derivativa do circuito. Esse circuito é largamente utilizado em laboratórios de física atômica ao redor do planeta \cite{florian:Thesis:2015}. Um sinal dispersivo gerado por espectroscopia é utilizado para alimentar o circuito, um sinal de controle gerado é utilizado para correção dos desvios. Com isso podemos estabilizar a frequência dos lasers experimentais com pouca variação de seu comprimento de onda.\\