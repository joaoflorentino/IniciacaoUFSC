\par Para este projeto, vou usar as ferramentas de software, como \emph{Matlab\tiny\textregistered}, que será usado nos cálculos matemáticos para equilibrar o equipamento \emph{PID} obtendo o melhor arranjo matemático para o mesmo, simulações do sistema em geral. Após determinado as melhores condições matemáticas para o sistema passa-se para a elaboração do circuito elétrico que pode ser projetado, montado e ensaiado no aplicativo \emph{Eletronic Workbench\tiny\textcopyright}, que desta forma evita danos em componentes elétricos e diminui os custos do projeto devido a estas ocorrências.

Assim que efetivado o melhor esquema elétrico do circuito elétrico e o mesmo tenha sido efetivamente montado e ensaiado em laboratório, passa-se a prototipagem da placa de circuitos e para tal, pode ser utilizado o Software \emph{Eletronics Rapid Prototyping - DesignSpark PCB 8.0\tiny\textcopyright} que elabora a placa de circuitos e deixa pronta para prototipagem.

Para documentação do trabalho estou usando o editor de textos para \LaTeX{} o \emph{Texmaker \tiny\textcopyright}.

