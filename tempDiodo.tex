Lasers baseados em semi-condutores (Diodos) são largamente utilizados em experimentos de física atômica, já que possuem funcionamento relativamente simples. Uma corrente elétrica é bombeada por um fotodiodo que emite luz, gerada pelo acoplamento dos elétrons da camada de valência à camada de condução. Esse processo gera luz em um largo intervalo de comprimentos de onda. Para a criação do Laser (\textit{Light Amplification by Stimulated Emission of Radiation}) é necessário que superfícies refletoras filtrem parte da luz emitida, retrorefletindo a maior parte da luz de volta para o diodo (cavidade do diodo). A adição de uma grade de difração externa a cavidade do diodo seleciona um comprimento de onda específico, que ao ser reinjetado na cavidade do diodo gera o fenômeno da inversão de população dos estados quânticos eletrônicos. O conjunto entre inversão de populações e da disponibilidade de fótons coerentes na cavidade faz com que o sistema apresente o chamado decaimento forçado (induzido) no mesmo comprimento de onda que os fótons de injeção. Esse sistema é conhecido como uma montagem ECDL (\textit{External Cavity Diode Laser}) emitindo luz num comprimento de onda quase puro, normalmente a largura de linha desse tipo de laser é de centenas de Kilohertz.\par
Contudo, o semi-condutor sofre forte influência das variações de temperatura do ambiente, já que a separação entre os níveis de energia entre as camadas de valência e condução são fortemente dependentes da temperatura, ou seja, se a temperatura muda, também se altera a frequência (energia) de emissão do laser. Para evitar essa variação um circuito eletrônico é utilizado, o sinal de erro é dado entre a temperatura alvo e a temperatura medida em tempo real da montagem. Um sistema PID estabiliza ativamente o sistema, fazendo com que o comprimento de onda permaneça estável, estando sob ação apenas de \textit{driftings} de longos intervalos temporais. \\
