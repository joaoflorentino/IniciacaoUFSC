O objetivo da iniciação científica do \emph{LabAtomo da UFSC},vinha a ser o desenvolvimento de um sistema eletrónico baseado em atuadores \emph{PID} para controle de temperatura do sistema emissor de \emph{LASER}. 

Inicialmente em um  projeto de circuito de controle \textbf{\emph{PID}}, há necessidade de se estabelecer uma equação de controle \textit{(Função de Transferência)}. Esta \emph{Função de Transferência} procura simular o comportamento do sistema a ser controlado matematicamente. No caso desta pesquisa, o sistema de controle, é basicamente o ajuste (resfriamento) de temperatura do \emph{Emissor de LASER} a ser utilizado no experimento, de acordo a descrição do projeto, e esta \emph{Função} pode ser representada por  uma equação linear que corresponde a evolução do aquecimento do sistema. 

Após a elaboração da \emph{Função de Transferência}, fez-se um estudo, utilizando o software \emph{Matlab\tiny\textregistered}, dos termos desta função com o objetivo de estabelecer os parâmetros de controle e simular diversas situações do sistema a ser implementado. 

Com a conclusão do estudo matemático do sistema, e obtenção dos parâmetros de controle P \emph{Proporcional}, I \emph{Integral} e D \emph{Diferencial}, o circuito eletrónico pode ser projetado.

O projeto iniciou-se baseado em um sistema que utilizasse circuitos discretos ( resistores, capacitores, indutores, diodos, amplificadores operacionais, etc ) para implementação do controle \emph{PID}. Com o desenvolvimento da mesma, outras possibilidades de circuitos tornaram-se viáveis. 

Durante as pesquisas de componentes a serem utilizados para implementação do circuito, encontrei um circuito que faz o papel de \emph{Drive} de pastilhas \textbf{\emph{Peltier}} e possibilita o controle de uma gama de temperaturas, tanto do lado quente quanto do lado refrigerado, deste elemento atuador, o circuito integrado \textbf{\emph{MAX 1968}} \emph{(Datasheet anexo)}. Internamente a este circuito integrado, ele possui um circuito \emph{PID} que gerência a temperatura controlando a mesma até o ponto de ajuste solicitado \textit{set point}. 

Para gerenciamento do circuito (ajuste de set point e leitura de sensores de temperatura) fiz a opção de utilizar um Microcontrolador  \textbf{PIC} \emph{(Peripherical Interface Controller) } que utilizando de poucos componentes externos e através de um programa de controle, que após desenvolvido, é gravado no microcontrolador, faria, no caso deste circuito, o controle dos displays de ajuste de temperatura, inserção do sinal no circuito \emph{MAX 1968} bem como a leitura dos sensores de temperatura . O esquema deste sistema pode ser visto na figura 11 deste trabalho.

Este trabalho de pesquisa, proposta pela iniciação cientifica, foi realizada no ano de 2017, e no inicio do ano de 2018 seria feita a implementação dos circuitos, testes e alterações que se fizessem necessárias. No entanto, o trabalho ficou parado nesse ponto da pesquisa, com o presente relatório e o circuito final não foi elaborado. 

Apesar do trabalho ter ficado (por falta de condições minhas) incompleto, foi uma grande oportunidade de aprendizado tanto no desenvolvimento de circuitos elétricos como na \textbf{\emph{Físca da UFSC}} que me trouxe um conhecimento sobre o futuro laboratório de \emph{Física Atómica} em desenvolvimento na UFSC que tive muito orgulho em ter tido a chance de participar. 

Agradeço também as orientações do \textbf{\emph{Jorge Douglas Massayuki Kondo, PHD}} e dos demais colegas ligados ao projeto. 

