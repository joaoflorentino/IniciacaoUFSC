Durante o projeto do sistema atuador de regulagem de temperatura \emph{PID}, objetivo da iniciação científica do \emph{LabAtomo} da UFSC, a pesquisa iniciou-se com a pesquisa baseada em um sistema que utilizasse circuitos discretos ( resistores, capacitores, indutores, diodos, amplificadores operacionais, etc ) para implementação do controle \emph{PID}.

Em um projeto de circuito de controle \textbf{\emph{PID}}, há necessidade de se estabelecer uma equação de controle. Esta \emph{equação de controle} procura simular o comportamento do sistema a ser controlado matematicamente. No caso de nosso sistema de controle, que vem a ser basicamente o controle de temperatura do LED LASER a ser utilizado no experimento, de acordo a descrição do projeto, vem a ser uma equação linear que corresponde a evolução do aquecimento do sistema. 

Após a elaboração da \emph{equação de controle}, o estudo foi feito utilizando o \emph{Matlab\tiny\textregistered}. Através deste software, pode-se estabelecer os parâmetros de controle e simular diversas situações do sistema a ser implementado. 

Durante as pesquisas de componentes a serem utilizados para implementação do circuito, encontrei um circuito que faz o papel de \emph{Drive} de pastilhas \textbf{\emph{Peltier}} e possibilita o controle de uma gama de temperaturas tanto do lado quente quanto do lado refrigerado do sistema. 
Internamente a este circuito integrado, ele possui um circuito \emph{PID} que gerencia a temperatura controlando a mesma até o ponto de ajuste solicitado \textit{set point}. 

Para gerenciamento do circuito fiz a opção de utilizar um Microcontrolador  \textbf{PIC} \emph{(Peripherical Interface Controller) } que utilizando de poucos componentes externos e através de um programa de controle, que após desenvolvido, é gravado no microcontrolador, faria, no caso deste circuito o controle dos displays de ajuste de temperatura e inserção do sinal no circuito \emph{MAX 1968}. O esquema deste sistema pode ser visto na figura 11 deste trabalho.

Até este ponto da pesquisa proposta pela iniciação cientifica a qual se deveu este trabalho foi realizada no ano de 2017, e para inicio do ano de 2018 seria feita a implementação dos circuitos, testes e alterações que se fizessem necessárias. No entanto, o trabalho ficou parado nesse ponto da pesquisa e o circuito final não foi elaborado. 

Apesar do trabalho ter ficado (por falte de condições minhas) incompleto, foi uma grande oportunidade de aprendizado tanto no desenvolvimento de circuitos elétricos como na \textbf{\emph{Físca da UFSC}} que me trouxe um conhecimento sobre o futuro laboratório de \emph{Física Atômica} em desenvolvimento na UFSC que tive muito orgulho em ter tido a chance de participar. 

Agradeço também as orientações do \textbf{\emph{Jorge Douglas Massayuki Kondo, PHD}} e dos colegas ligados ao projeto. 


