\documentclass[a4paper, 12pt]{article}
\usepackage{geometry}
\geometry{paperwidth=210mm,paperheight=297mm,textwidth=160mm, textheight=210mm,top=30mm,bottom=20mm,left=30mm,right=20mm}
\usepackage[utf8]{inputenc}
\usepackage[brazil] {babel}
\usepackage{amsmath}
\usepackage{amsfonts}
\usepackage{amssymb}
\usepackage{import}
\usepackage{multicol}
\usepackage{multirow}
\usepackage{verbatim}
\usepackage{tikz}
\usepackage{graphicx,xcolor}
\usepackage{xwatermark}
\usepackage{pgfplots}
\pagenumbering{arabic}
\usepackage{float}
\usepackage{setspace}
\setlength{\parindent}{2cm}
\usepackage{hyperref}
%pacote para desenho de diagramas de bloco e etc.
\usepackage{tikz}
\hypersetup{
    colorlinks=true,
    linkcolor=blue,
    filecolor=magenta,      
    urlcolor=cyan,
}
\urlstyle{same}

\usepackage{ragged2e}
\usepackage{tabularx,colortbl}
\usepackage{cite}
\usepackage{tikz} 
% Inserção de Arquivos ANEXOS
\usepackage{pdfpages}

% Entrada de Marca D'agua
%\newwatermark[allpages,angle=60,scale=2,color=blue!20]{J L C F}


%%%% Tentativa 7 joao
\title{
		\RaggedRight 
		\vspace{-0.25in} 	
		\usefont{OT1}{bch}{b}{n}
		\normalfont \normalsize \textsc{\textbf{Projeto de Iniciação Científica}
		 (\textcolor{olive}{\textbf{Universidade Federal de Santa Catarina}})} \\ [10pt]
		Período: \textcolor{blue}{2017-2/2018-1}\\
		\huge Exploração e montagem de um circuito de controle de Feedback ativo para uso no laboratório de Física atômica (ATOMO LAB)
}
\author{		
		\normalfont \normalsize
		\RaggedRight
        Docente orientador: Prof. Dr. Jorge Douglas Massayuki Kondo \\
        Orientando: João Luis Calmon Florentino
}
\author{
\RaggedRight
Departamento: Departamento de Física\\
Unidade: Centro de Ciências Físicas e Matemáticas (CFM) - UFSC\\
Página eletrônica: \url{www.atomobrasil.com}\\
Docente orientador: \textbf{\emph{Prof. Dr. Jorge Douglas Massayuki Kondo}}\\
E-mail: massayuki.kondo@ufsc.br\\
Aluno orientando:  \textbf{\textit{João Luis Calmon Florentino}}\\
Física - Bacharelado UFSC \\
Matrícula número 17150281\\
E-mail:joao.florentino.senai@gmail.com\\
}

\date{Data: \today}



\begin{document}
%\RaggedLeft
%\justifying
\centering
	\begin{figure}
		\centering
		\includegraphics[width=0.3\linewidth]{./ima/brasaoUFSC}
		%\caption{}
		\label{fig:brasaoufsc}
	\end{figure}

\RaggedRight
\maketitle

\newpage

\tableofcontents \newpage
%\pagebreak
\section{Resumo}
\import{./}{resumo1}

\section{Travamento em frequência de lasers}
\import{./}{travamento1}

\section{Controle de Temperatura de laser de Diodo}
\import{./}{tempDiodo}

\section{Cronograma do plano de trabalho}
\import{./}{cronograma1}

\section{Desenvolvimento}
\import{./}{desenv1}

\subsection{Objetivos}
\import{./}{objet1}

\subsection{Função de Transferência}
\import{./}{ftrsf01}

\subsection{Técnicas de elaboração de projeto}
\import{./}{tecpro01}

	\subsubsection{Simulações Matlab Simulink }
	\import{./}{simulink1}
	
\subsection{Sensores e Atuadores}
\import{./}{sensoresAtuadores1}

	\subsubsection{Considerações Teóricas - Calor }
	\import{./}{calor1}
	\subsubsection{Transferencia de Calor}
	\import{./}{transferCalor1}

\subsection{Determinação das Funções de Transferência}
\import{./}{equaPeltier1}

\subsection{Determinação das constantes de ganho do PID}
\import{./}{analmatlab1}

% capitulo de Montagem do Circuito - pos-estudos de projeto
\section{Montagem do projeto}
\import{./}{monte01}

\subsection{Esquema e PCB do circuito PID}
\import{./}{esquema1}

\subsection{Especificações dos Componentes}
\import{./}{componentes1}

\section{Conclusão}
\import{./}{conclusao1}

\newpage

%Referencias do projeto 
\bibliographystyle{unsrt}
\bibliography{ref0A}{}

%ANEXOS 
\newpage
% % \section{ANEXOS}
\import{./}{anexos1}



\end{document} 